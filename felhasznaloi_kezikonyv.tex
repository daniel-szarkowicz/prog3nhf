\documentclass{article}

\usepackage[a4paper, margin=1in]{geometry}
\usepackage{parskip}

\title{Pac-Man felhasználói kézikönyv}
\author{Szarkowicz Dániel - FK0IEH}
\date{}

\begin{document}

\maketitle

\section{Főmenü}

A főmenüben négy gomb van: Play, Load, Edit és Exit.

A Play gombra nyomva lehet kiválasztani egy pályát, majd a játékosok számát.\\
A Load gombra nyomva lehet betölteni egy előzőleg kimentett játékot.\\
Az Edit gombra nyomva lehet megnyitni a pályaszerkesztőt.\\
Az Exit gombra nyomva lehet kilépni a játékból.

\section{Játék}

A játékosok irányítják a Pac-Man-eket, a céljuk a szörnyek elkerülése és a
pontok összegyűjtése. Négy szörny van: Inky (kék), Clyde (narancssárga),
Blinky (piros) és Pinky (pink).

Inky teljesen véletlenszerűen mozog a labirintusban.\\
Clyde véletlenszerűen mozog a labirintusban, de ha egy Pac-Man közel kerül
hozzá, akkor elkezdi üldözni.\\
Blinky az útjukon követi a Pac-Man-eket. Mindig a legfrissebb úton halad,
amíg nem talál egy utat, addig véletlenszerűen mozog.\\
Pinky a legrövidebb úton üldözi a Pac-Man-eket.

A pályán kisebb és nagyobb pontok vannak. A nagyobb pontok 5 másodpercre
felerősítik azt a Pac-Man-t amelyik megette, az erősebb Pac-Man gyorsabban
mozog és meg tudja enni a szörnyeket.

Az ablak tetején egy menü található.\\
A File > Save opcióval lehet kimenteni a játék jelenlegi állapotát.\\
A File > Exit opcióval lehet kilépni a főmenübe.

\section{Pályaszerkesztő}

A pályaszerkesztővel lehet új pályákat készíten, vagy régebbi pályákat
módosítani.

A File > Open opcióval lehet egy pályát megnyitni.\\
A File > Save opcióval lehet a jelenlegi pályát kimenteni.\\
A File > New opcióval lehet egy új pályát készíteni, ekkor ki lehet választani
a pálya szélességét és magasságát.\\
A File > Exit opcióval lehet kilépni a főmenübe.

Az Edit > Wall opcióval lehet a falakat ki vagy be kapcsolni.\\
Az Edit > Pacman Spawn opcióval lehet a Pac-Man-ek kezdőhelyét módosítani.\\
Az Edit > Monster Spawn opcióval lehet a szörnyek kezdőhelyét módosítani.\\
A két fél kezdőhelye nem lehet a falakon és ugyan azon a helyen.

\end{document}