\documentclass{article}

\usepackage[a4paper, margin=1in]{geometry}
\usepackage{parskip}
\usepackage[hungarian]{babel}
\usepackage{hyperref}
\usepackage{tabularray}

\hypersetup{
	colorlinks,
	linkcolor=blue
}

\title{Pac-Man felhasználói kézikönyv}
\author{Szarkowicz Dániel - FK0IEH}
\date{}

\begin{document}

\maketitle

\section{Főmenü}

A főmenüben négy gomb van: Play, Load, Edit és Exit.

A \textbf{Play} gombra nyomva lehet kiválasztani egy pályát, majd a játékosok számát.\\
A \textbf{Load} gombra nyomva lehet betölteni egy előzőleg kimentett játékot.\\
Az \textbf{Edit} gombra nyomva lehet megnyitni a pályaszerkesztőt.\\
Az \textbf{Exit} gombra nyomva lehet kilépni a játékból.

\section{Játék}

A játékosok \hyperref[appendix:controls]{irányítják} a Pac-Man-eket, a céljuk
a szörnyek elkerülése és a pontok összegyűjtése.
Négy szörny van: Inky (kék), Clyde (narancssárga), Blinky (piros) és Pinky (rózsaszín).

\textit{Inky} teljesen véletlenszerűen mozog a labirintusban.\\
\textit{Clyde} véletlenszerűen mozog a labirintusban, de ha egy Pac-Man közel kerül
hozzá, akkor elkezdi üldözni.\\
\textit{Blinky} az útjukon követi a Pac-Man-eket. Mindig a legfrissebb úton halad,
amíg nem talál egy utat, addig véletlenszerűen mozog.\\
\textit{Pinky} a legrövidebb úton üldözi a Pac-Man-eket.

A pályán kisebb és nagyobb pontok vannak. A nagyobb pontok 5 másodpercre
felerősítik azt a Pac-Man-t amelyik megette, az erősebb Pac-Man gyorsabban
mozog és meg tudja enni a szörnyeket.

Az ablak tetején egy menü található.\\
A \textbf{File > Save} opcióval lehet kimenteni a játék jelenlegi állapotát.\\
A \textbf{File > Exit} opcióval lehet kilépni a főmenübe.

\section{Pályaszerkesztő}

A pályaszerkesztővel lehet új pályákat készíten, vagy régebbi pályákat
módosítani.

A \textbf{File > Open} opcióval lehet egy pályát megnyitni.\\
A \textbf{File > Save} opcióval lehet a jelenlegi pályát kimenteni.\\
A \textbf{File > New} opcióval lehet egy új pályát készíteni, ekkor ki lehet választani
a pálya szélességét és magasságát.\\
A \textbf{File > Exit} opcióval lehet kilépni a főmenübe.

Az \textbf{Edit > Wall} opcióval lehet a falakat ki vagy be kapcsolni.\\
Az \textbf{Edit > Pacman Spawn} opcióval lehet a Pac-Man-ek kezdőhelyét módosítani.\\
Az \textbf{Edit > Monster Spawn} opcióval lehet a szörnyek kezdőhelyét módosítani.\\
A két fél kezdőhelye nem lehet a falakon és ugyan azon a helyen.

\newpage
\appendix

\section{Pac-Man irányítása}\label{appendix:controls}

Pac-Man négy irányba mozoghat: fel, le, jobbra és balra. Pac-Man a következő
lehetséges ponton fog elfordulni a megadott irányba, addig a jelenlegi
iránya szerint fog tovább mozogni.

A játékosok különböző billentyűkkel irányítják a Pac-Man-eket:

\begin{tblr}{hlines, vlines, row{1}={cmd=\textbf}, column{1}={cmd=\textbf}}
Játékos & szín    & fel        & le           & jobbra        & balra       \\
1.      & sárga   & W          & S            & D             & A           \\
2.      & zöld    & $\uparrow$ & $\downarrow$ & $\rightarrow$ & $\leftarrow$\\
3.      & magenta & I          & K            & L             & J           \\
4.      & türkiz  & T          & G            & H             & F           \\
\end{tblr}

\end{document}